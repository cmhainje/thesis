\chapter{Introduction}
\label{sec:intro}

\section{Dark matter}
One of the most important unsolved problems in physics is the missing mass
problem~\cite{trimble_existence_1987}. It was first identified by Oort while studying galactic rotation
curves, where he found that the rotation curve increased linearly with
radius, an indication of a constant mass density, despite the fact that the
amount of visible matter fell off. The same trend was found in many other
galaxies over the following years, indicating that the missing mass problem
was not specific to the galaxy Oort was studying.

The peculiar rotation curves led researchers to explore a few possible
solutions~\cite{tucker_mystery_1984}. The first: perhaps our theory of
gravity fails on galactic scales and needs modification. This approach would
turn out to fail in large part because it requires modifying general
relativity in a manner that remains consistent at \textit{all} scales. The
second: perhaps there is some other force at work that modifies the galactic
dynamics to produce this phenomenon. This again would be discarded, as there
is no other evidence for such a force to exist. The third, and widely
considered to be ``true'', solution is that there is some form of unseen
matter about the galaxies that keeps the mass density constant with
increasing radius. Because this matter does not interact with the standard
model in a way that allows it to be seen, it is known as ``dark matter''
(DM).

As time has continued, evidence for dark matter has been found in a number of
different astrophysical and cosmological data. One such phenomenon is
gravitational lensing, whereby a large distribution of matter can bend the
path of light in a measurable way. Observed gravitational lensing caused by
galaxies and galaxy clusters is in general stronger than would be expected
from visible matter alone, but is consistent with the existence of dark
matter in the same quantities as predicted by the rotation
curves~\cite{schneider_gravitational_1992}. These and other cosmological data
have converged on a ``standard model of cosmology'' which includes the
existence of dark matter as an important component of the universe.

\section{Standard cosmological model}
The standard cosmological model is the $\Lambda$CDM model. This model treats
the universe as consisting primarily of three components: the cosmological
constant $\Lambda$ describing the energy density of vacuum, cold dark matter
(CDM), and ordinary matter. Of chief relevance for our study is the CDM
component, which this model takes to be ``cold''---moving at speeds much
slower than the speed of light---and collisionless---interacting with itself
and other particles through gravity alone.

The $\Lambda$CDM model is known as the standard cosmological model because it
is a simple model able to account for many of the dominating cosmological
phenomena. In particular, its predictions of the existence of the cosmic
microwave background agree with observation, it easily accounts for the
accelerating expansion of the universe, and it can explain the hierarchical
large-scale ($\gtrsim \mathcal{O}(\text{Mpc})$) structures of the
distribution of galaxies, like galaxy clusters, superclusters, etc. Despite
these strengths, however, one of its largest discrepancies with observation
occurs in the small-scale structures of galaxies.

There are several \textit{small-scale problems} that have been discovered in
high resolution simulations of the $\Lambda$CDM paradigm. Some of the most
prominent ones are discussed below, following the discussion
of~\cite{tulin_dark_2018}.
\begin{itemize}
    \item The \textit{core-cusp problem}: in simulation, CDM halos generally
    evolve to have a ``cuspy'' mass density profile---one which goes like $1/r$
    (for $r$ the radius) near the galactic center. The observed rotation curves
    of many galaxies, however, indicate ``cored'' mass density profiles---ones
    which are constant with radius near the center. 

    \item The \textit{missing satellites problem}: CDM halos evolve from the
    mergers of small halos and thus should have a rich substructure. As such,
    simulations can be used to obtain predictions about the substructure and, in
    particular, the number of subhalos (satellite galaxies) that should be
    present in and about galaxies of certain sizes. However, observations of the
    Milky Way show that there are far fewer satellite galaxies than are
    predicted by simulation.

    \item The \textit{diversity problem}: CDM models predict a remarkably
    self-similar process for structure formation, indicating that halos of a
    similar mass should have similar structure. Observations, however, show that
    many disk galaxies with similar halo mass have relatively large
    discrepancies in their interior structure and core densities.

    \item The \textit{too-big-to-fail problem}: The brightest satellite galaxies
    in the Milky Way are expected to host the most massive subhalos. However,
    CDM simulations predict the most massive subhalos for a galaxy the size of
    the Milky Way to be far too dense to be consistent with the stellar dynamics
    of observable satellite galaxies. However, subhalos of the predicted mass
    and density are ``too big to fail'' in forming observable galaxies, leaving
    us to wonder what happened to these massive subhalos.
\end{itemize}

There are a few possible alternatives to the $\Lambda$CDM model that are being
investigated. These alternatives generally involve considering different forms
of dark matter to account for these discrepancies in the small-scale structure.
Some researchers believe that the culprit is that most major CDM simulations
are DM-only, meaning that the small-scale structure would be validated by the
inclusion of baryonic processes. However, many believe that baryonic processes
are not significant enough to solve this problem. Another possibility is the
consideration of ``warm'' DM, which yields interesting predictions for the
influences of dark matter at earlier times in the universe. A third possiblity,
and the one with which we primarily concern ourselves for this study, is that
the DM may have a mechanism for self-interaction. This possibility will be
covered in more detail in Chapter~\ref{sec:sidm}.

\section{Sagittarius stream}
In this paper, we largely consider the Sagittarius (Sgr) satellite galaxy of
the Milky Way. Sgr is a dwarf spheroidal (dSph) galaxy orbiting the MW. It is
a ``dwarf'' galaxy due to its low absolute magnitude, a measure of its
brightness, and it is ``spheroidal'' due to its luminosity (opposed to
``elliptical'')~\cite{mcconnachie_observed_2012}. It was first discovered by
Ibata et al. in 1994~\cite{ibata_dwarf_1994}.

Since its discovery, Sgr has been the subject of many studies due to a few
interesting features of its past. It is one of the nearest dwarf galaxies in
the MW, making it somewhat easier to obtain detailed measurements
for~\cite{kunder_distance_2009}. It is also widely believed to have made
several passages through the MW disk over its
orbits~\cite{purcell_sagittarius_2011}. This fact in particular makes it
greatly interesting, as these orbits have resulting in the stripping of stars
from Sgr by the tidal forces of MW gravity, leaving a stream of Sgr debris
which wraps the MW completely~\cite{purcell_sagittarius_2011,
law_sagittarius_2010}.

The evolution of the Sgr stream is evidently quite sensitive to changes in
the MW gravitational potential, the Sgr progenitor mass, and other properties
of the galaxies. As such, its study could prove very insightful for obtaining
a better understanding of these properties. It is difficult, however, to
reconstruct the Sgr orbital history, because different histories are possible
when one varies the Sgr mass~\cite{jiang_orbit_2000}. The most prominent
models have been those due to Law \& Majewski~\cite{law_sagittarius_2010} and
Purcell et al.~\cite{purcell_sagittarius_2011}. More recently, however, a
high resolution simulation study was performed that used ``live'', i.e.
dynamic, gravitational potentials for \textit{both} MW and
Sgr~\cite{dierickx_predicted_2017}. This study is the one on which we base
the majority of our simulation work.

One unexplored parameter that would necessarily impact the evolution of the Sgr
stream is the DM model used in the simulation of the halos of the Sgr and MW
galaxies. Our aim in this study is to analyze and describe the differences
between the Sgr stream evolution using a CDM model---the only model yet
considered---and self-interacting DM models. 
