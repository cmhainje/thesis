\section{SIDM simulations}
\label{sec:sidm-sim}

\textit{%
    Note to reader: at this stage in my research, I have not yet completed SIDM
    simulations, owing primarily to a number of complications that have occurred
    with GIZMO. As such, in this and the following sections I will only
    describe the analysis that I intend to do.
}

\subsection{Initial galaxy equilibration}
The SIDM simulations are performed using much the same pipeline as with the CDM
simulations. In particular, we create the individual MW and Sgr galaxies and
evolve each independently for 5-10 Gyr to ensure equilibration before merging.
The halo and disk particles are drawn from the same mass distributions as
discussed previously.

For SIDM, the initial equilibration evolution is more important to track than
for CDM, as the introduction of self-interaction with too high or low a cross
section could strongly disturb the stability of our galaxies. We begin with a
cross section of around $\sigma / m = 1$ cm$^2$/g, similar to that discussed
in Section~\ref{sec:sidm}. The evolution of the halo and disk density
distributions will be given in a Figure, alongside a discussion of their
implications.

\subsection{Sgr infall evolution}
With a well-tuned cross section and equilibrated individual galaxies, we merge
the galaxies as was done previously and evolve them for 10 Gyr. 

Here will follow a discussion of results similar to those discussed for CDM
(evolution of the MW mass distributions, trajectory of the Sgr center of mass,
evaluation of observable coordinates, and illustrations of the Sgr DM/star
particles). Further, I intend to measure quantitative properties of the Sgr
stream (e.g. velocity dispersion, etc.) and compare to the CDM results.

