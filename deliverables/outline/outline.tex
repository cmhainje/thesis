\documentclass[12pt]{article}
\usepackage{lmodern}
\usepackage[margin=1in]{geometry}

\renewcommand{\labelenumi}{\arabic{enumi}}
\renewcommand{\labelenumii}{\labelenumi.\arabic{enumii}}
\renewcommand{\labelenumiii}{\labelenumii.\arabic{enumiii}}

\begin{document}

% Custom title
{
\centering
\Large \textbf{%
    Dependence of Sagittarius tidal stream properties on dark matter model
} \par
\vspace{10pt}
\normalsize Connor Hainje \par
\vspace{4pt}
\textit{Advisor:} Mariangela Lisanti \par
}


\section*{Abstract}
The Sagittarius (Sgr) tidal stream is a large, complex structure of stars from
the Sagittarius dwarf galaxy which wraps around the Milky Way. Precisely
characterizing its kinematics can constrain the properties of the Sgr and Milky
Way halos, providing information about the properties of the dark matter
therein. We use $N$-body simulation methods to simulate the entire infall of
Sgr, modeling the dark matter halos using CDM and SIDM models. We compare the
resulting predicted tidal streams to determine the relationship between dark
matter models and the tidal stream structure.  Finally, we compare these results
to Sgr stream data.


\section*{Outline}
\begin{enumerate}

\item Introduction
\begin{enumerate}
    \item Dark matter models
    \begin{enumerate}
        \item Cold dark matter
        \item Self-interacting dark matter
    \end{enumerate}

    \item Sgr stream
\end{enumerate}

\item $N$-body simulation
\begin{enumerate}
    \item Parameters and initial conditions
    \item Overview of simulation
    \item Simulation results
    \begin{enumerate}
        \item Cold dark matter
        \item Self-interacting dark matter
    \end{enumerate}
\end{enumerate}

\item Analysis
\begin{enumerate}
    \item Impact of dark matter model on tidal stream
    \item Comparison to Sgr data
\end{enumerate}

\item Summary and conclusions

\end{enumerate}

\end{document}
