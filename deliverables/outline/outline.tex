\documentclass{article}
\usepackage{lmodern}

% Change numbering
\renewcommand{\labelenumi}{\arabic{enumi}}
\renewcommand{\labelenumii}{\labelenumi.\arabic{enumii}}
\renewcommand{\labelenumiii}{\labelenumii.\arabic{enumiii}}

\begin{document}

% Custom title
{
\centering
\Large \textbf{%
    Dependence of Sagittarius stream properties on dark matter model
} \par
\vspace{12pt}
\normalsize Connor Hainje \par
\vspace{3pt}
\textit{Advisor:} Mariangela Lisanti \par
}


\section*{Abstract}
The Sagittarius (Sgr) tidal stream is a large structure of stellar debris from
the Sagittarius dwarf galaxy which has been torn away by the Galactic tidal
forces of Milky Way (MW). This stream wraps at least once around the Milky Way,
placing many Sgr stars in the MW halo. The kinematics by which the Sgr stream
has evolved depend both on stellar properties of the MW and Sgr galaxies, but
also on their dark matter halos. We aim to determine how the evolution of the
Sgr streams depends on the dark matter model used to characterize the Sgr and MW
halos. To do so, we use $N$-body simulation methods to simulate the entire
infall of Sgr, modeling the dark matter halos using both cold and
self-interacting dark matter models. We compare the resulting predicted tidal
streams to determine the relationship between dark matter models and the tidal
stream structure.  Finally, we compare these results to Sgr stream data.


\section*{Outline}
\begin{enumerate}

\item Introduction
\begin{enumerate}
    \item Dark matter models
    \begin{enumerate}
        \item Cold dark matter
        \item Self-interacting dark matter
    \end{enumerate}

    \item Sgr stream
\end{enumerate}

\item $N$-body simulation
\begin{enumerate}
    \item Parameters and initial conditions
    \item Overview of simulation
    \item Simulation results
    \begin{enumerate}
        \item Cold dark matter
        \item Self-interacting dark matter
    \end{enumerate}
\end{enumerate}

\item Analysis
\begin{enumerate}
    \item Impact of dark matter model on tidal stream
    \item Comparison to Sgr data
\end{enumerate}

\item Summary and conclusions

\end{enumerate}

\end{document}
