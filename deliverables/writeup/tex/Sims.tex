\hypertarget{full-infall-simulations}{%
\chapter{Full infall simulations}\label{full-infall-simulations}}

\hypertarget{description-and-initial-results}{%
\section{Description and initial
results}\label{description-and-initial-results}}

With the equilibrated and merged initial conditions for both cuspy (CDM) and
cored (SIDM) galaxies, we now carry out our full simulations of the Sgr
infall.  We will consider \emph{three} mergers: the cuspy initial conditions
evolved using CDM microphysics, the cored initial conditions evolved with CDM
microphysics, and the cored initial conditions evolved with SIDM microphysics.
As before, we take \(\sigma / m\) = 10 cm\(^2\)/g in the SIDM case.  These
three mergers will be referred to as CDM/cusp, CDM/core, and SIDM
respectively.  By performing all three simulations, we will ideally be able to
identify whether certain discrepancies between the CDM/cusp and the SIDM runs
are the result of a cored initial profile or from the inclusion of
self-interactions.

For each merger, the infall is simulated for 10 Gyr, with snapshots saved
every 0.978 Gyr.  In Figure todo, we show the positions of the stars of Sgr in
the orbital plane at several times for each merger.  Similarly, we show the
positions of the Sgr dark matter particles in Figure todo.

todo Figure: sgr stars for all three mergers at t = 0, 3, 6, 9

todo Figure: same but with DM

Even from these plots, some differences appear to emerge.

todo talk more about this

\hypertarget{identifying-the-sgr-progenitor}{%
\section{Identifying the Sgr
progenitor}\label{identifying-the-sgr-progenitor}}

A key part of analyzing these data is to understand the trajectory and
evolution of the Sgr progenitor in particular. As such, we desire a
method for successfully identifying the position of the Sgr progenitor
throughout its evolution. This is less straightforward than it may sound
because of the strong effects of tidal stripping. These mean that we
need to identify which particles are stripped or bound to the progenitor
at any given point and omit those particles which have been stripped
from our calculation of the progenitor position. In our tests, we tried
a few different methods which we will describe here.

The first method that we tried was to track bound versus unbound star
particles by counting particles as stripped once they exceeded a fixed radius
from the center of mass of those which are still bound.  The algorithm for
this is as follows.  We begin by counting the stellar particles within a
certain radius on the first snapshot to be ``bound''.  For each snapshot
after, we find the center of mass of the particles which are bound.  For each
bound particle, we compute its distance from the center of mass.  If this
exceeds the fixed stripping radius, we unmark the star as bound and continue.

As stated, this algorithm leaves has two parameters that can be tuned: the
initial stripping radius for the initial Sgr stellar positions and the fixed
stripping radius for all following snapshots. We found it useful to describe the
initial stripping radius instead in terms of the percentage of particles
that are initial counted as ``bound''. For example, we say that the we start
with the innermost 20\% of particles and proceed with a fixed radius of 20 kpc. 
effective for our mergers. The results of applying this algorithm to the
CDM/Cusp merger data with a few different choices of parameters can be seen in
Figure~\ref{fig:fixed_star}.

\begin{figure}
    \centering
    \includegraphics[width=0.9\linewidth]{figs/fixed_star.pdf}
    \caption{%
        Results of applying the ``fixed stripping radius''
        progenitor-identifying algorithm to the CDM/cusp merger data. Entries in
        the legend are given in the following format: ``a/b'' means that we
        started with the innermost ``a''\% of stellar particles and proceeded
        with a fixed stripping radius of ``b'' kpc.  In the upper left is the
        number of bound particles over time.  The upper right shows the
        stripping radius.  The bottom left shows the distance from the origin
        to the Sgr center of mass; an estimate of the MW-Sgr separation.  The
        bottom right shows the trajectory of the progenitor in the orbital
        plane.
    }
    \label{fig:fixed_star}
\end{figure}

This figure shows us that starting with too few of the initial particles (in
this case, 20\%) leads to very small numbers of bound particles at late times.
Further, it shows us that the trajectory of the progenitor can be somewhat
sensitive to the chosen algorithm parameters, especially at late times.

This algorithm appears to have two issues that we want to try to solve.
First, the actual size of the progenitor is expected to shrink with time, as
progressively more of the particles are stripped.  By using a fixed stripping
radius, we are not modeling the expected decay of the progenitor size.  The
second problem we encountered is that this method appears to leave us with
only $\mathcal{O}(10)$ bound particles after around 6 Gyr evolved.  As such,
we decided to explore modifications to the algorithm.

The first modification was to consider a decreasing stripping radius.  The
algorithm is very similar to before.  On the first snapshot, we count some
inner fraction of the particles to be ``bound'' and find the radius of this
ball of bound particles.  For the next snapshot, we strip any particles
which exceed this radius (times a constant).  \textit{However}, after
stripping away particles, we recompute the radius of the ball of the bound
particles, and reset the stripping radius equal to this. For each snapshot
following, we strip particles that are beyond this radius and recompute the
radius.  Over time, the stripping radius will decrease, modeling the
progressively decreasing size of the progenitor.  However, these modifications
tend to make the algorithm a bit \textit{too} aggressive, with it stripping
away all the particles in some cases.  We introduced a minimum stripping
radius such that the algorithm would choose the maximum of the given minimum and
the computed radius, but we found that the behavior in this case reduced to that
of the fixed-radius algorithm. The results of this decaying stripping radius are
shown in Figure~\ref{fig:decay_star}, with a minimum allowed stripping radius
of 8 kpc.

\begin{figure}
    \centering
    \includegraphics[width=0.9\linewidth]{figs/decay_star.pdf}
    \caption{%
        Results of applying the ``decaying stripping radius''
        progenitor-identifying algorithm to the CDM/cusp merger data with a
        minimum stripping radius of 8 kpc.  Plots have same meaning as in
        Figure~\ref{fig:fixed_star}.
    }
    \label{fig:decay_star}
\end{figure}

As can be seen in the figure, this algorithm leads to very aggressive stripping.
No matter the initial conditions, the desired stripping radius falls
\textit{very} quickly. For all considered parameter combinations, the desired
stripping radius fell below 8 kpc before the 6 Gyr mark, at which point the
algorithm will just default to using a fixed 8 kpc stripping radius. This
indicates that these changes are simply too aggressive and, when bundled with a
minimum stripping radius, is effectively equivalent to the fixed stripping
radius scheme.

One possible reason that it becomes too aggressive is because the actual
size of the progenitor is not monitonically decreasing.  Rather, its size
fluctuates over the course of the orbit, becoming quite compressed and small
near the pericenter and a bit more spread out and large near the apocenter.
These effects are modeled by the King formula for the tidal radius (todo cite
King) as given in~\cite{dierickx_predicted_2017}:
\begin{equation} \label{eq:king_radius}
    r_t = r \left[ \frac{1}{2} 
    \frac{M_{\text{Sgr}(<r_t)}}{M_{\text{MW}(<r)}} \right]^{1/3},
\end{equation}
where $M_{\text{gal}}(<r)$ is the enclosed halo mass in galaxy ``gal'' within
radius $r$ of the center of mass of the galaxy, $r$ is the distance between
the Milky Way and Sgr centers of mass, and $r_t$ is the tidal radius.  For a
given snapshot, then, we can compute the tidal radius according to this
formula by subtracting $r_t$ from both sides and using a simple root finder to
identify the $r_t$ which solves the equation.  We then strip any particles
which are farther than $r_t$ away from the center of mass of the progenitor at
the current time. The results of using the King tidal radius are shown in
Figure~\ref{fig:king_star}. Again, we pair this algorithm with a minimum
stripping radius of 8 kpc.

\begin{figure}
    \centering
    \includegraphics[width=0.9\linewidth]{figs/king_star.pdf}
    \caption{%
        Results of using the ``King tidal radius'' progenitor-identifying
        algorithm on the CDM/cusp merger data with a minimum stripping radius of
        8 kpc. The legend entries correspond to a fixed constant multiplied
        against the tidal radius before performing stripping. Plots again have
        the same meaning as in Figure~\ref{fig:fixed_star}.
    }
    \label{fig:king_star}
\end{figure}

Yet again, this algorithm appears to simply be too aggressive. In all cases, the
tidal radius quickly dips below the minimum stripping radius of 8 kpc. Further,
the algorithm appears to be somewhat unstable, defaulting to a recommended
radius of just \textit{zero} beyond a certain threshold. When this occurs, the
algorithm defaults to a fixed stripping radius equal to the minimum. Again,
then, this algorithm offers no improvements over the fixed radius.

At this point, it seems like the issue of too few bound particles at late times
may be a problem related to our relatively small number of Sgr stellar particles
instead of a problem with the algorithms themselves.  As such, we decided to
try to use the fixed algorithm but using both stellar \textit{and} dark matter
particles.  The result on the CDM/cusp data is shown in
Figure~\ref{fig:fixed_both}.

\begin{figure}
    \centering
    \includegraphics[width=0.9\linewidth]{figs/fixed_both.pdf}
    \caption{%
        Results of using the ``fixed stripping radius'' progenitor-identifying
        algorithm on the all particles in the CDM/cusp merger data. Plots and
        legend entries have the same meaning as in
        Figure~\ref{fig:fixed_star}.
    }
    \label{fig:fixed_both}
\end{figure}

This appears to yield promising results. We note that there are generally more
``bound'' particles at late times when using all particles than when only using
stellar ones, and that, aside from the ``50/20'' and ``80/20'' runs, the
resulting trajectories appear to be more robust to the algorithm parameters. As
such, we choose to move forward with this algorithm using the ``20/20''
parameters, as they appear to be consistent with the majority of the other
parameter choices and yield the most bound particles in the end. Using these
choices to identify the progenitor, we apply the algorithm to all three mergers.
The resulting data are shown in Figure~\ref{fig:all}.

\begin{figure}
    \centering
    \includegraphics[width=0.9\linewidth]{figs/all_mergers_pretty.pdf}
    \caption{%
        Results of using the ``fixed stripping radius'' progenitor-identifying
        algorithm on all the particles in each of the mergers. We use the
        ``20/20'' algorithm parameters for all mergers, meaning that we start
        with the inner 20\% of particles and use a 20 kpc stripping radius. The
        top row shows the trajectory of the Sgr progenitor in the orbital plane
        for each merger. The middle row plot shows the number of bound particles
        over time. The bottom row plot shows the MW-Sgr separation over time.
    }
    \label{fig:all}
\end{figure}

todo discuss what this plot means!

\hypertarget{comparison-to-stream-data}{%
\section{Comparison to stream data}\label{comparison-to-stream-data}}

comparison to streamfinder data
