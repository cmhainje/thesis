\hypertarget{full-infall-simulations}{%
\chapter{Full infall simulations}\label{full-infall-simulations}}

\hypertarget{description-and-initial-results}{%
\section{Description and initial
results}\label{description-and-initial-results}}

With the equilibrated and merged initial conditions for both cuspy (CDM) and
cored (SIDM) galaxies, we now carry out our full simulations of the Sgr
infall.  We will consider \emph{three} mergers: the cuspy initial conditions
evolved using CDM microphysics, the cored initial conditions evolved with CDM
microphysics, and the cored initial conditions evolved with SIDM microphysics.
As before, we take \(\sigma / m\) = 10 cm\(^2\)/g in the SIDM case.  These
three mergers will be referred to as CDM/cusp, CDM/core, and SIDM
respectively.  By performing all three simulations, we will ideally be able to
identify whether certain discrepancies between the CDM/cusp and the SIDM runs
are the result of a cored initial profile or from the inclusion of
self-interactions.

For each merger, the infall is simulated for 10 Gyr, with snapshots saved
every 0.978 Gyr.  In Figure~\ref{fig:inits}, we show the positions of the
both the stellar and dark matter particles of Sgr in the orbital plane at
several times for each merger.

\begin{figure}
    \centering
    \includegraphics[width=0.9\linewidth]{figs/stars_bw.png}
    \includegraphics[width=0.9\linewidth]{figs/darks_bw.png}
    \caption{%
        Positions of Sgr stellar (top three rows) and dark matter (bottom
        three rows) particles at various times for all three of the considered
        mergers.  Each column denotes a different time; each row a different
        merger.
    }
    \label{fig:inits}
\end{figure}

Even from these plots, there are some interesting patterns to note. First, we
notice the development of a winding stream structure much like that reported
in~\cite{dierickx_predicted_2017} by around 6 to 7 Gyr in all three cases. We
can also note that the CDM/core and CDM/cusp are rather similar. There are some
discernible differences, but the general shape and size of the stream structure
is fairly similar throughout. 

There are quite marked differences between the SIDM and CDM cases, however.
The stream arms appear to be slightly rotated clockwise relative to the CDM
mergers, and the general shape of the inner structure is very different. These
differences are even more apparent when looking at the distribution of dark
matter. In the CDM mergers, the dark matter distribution appears to roughly
trace out the distribution of the stream and has a distinct hole at the
origin.  In the SIDM merger, however, the dark matter distribution largely
loses the shape of the stream, instead collapsing inward toward the MW center.

We can also look at the density of particles in the orbital plane by performing
a two-dimensional histogram on the data in Figure~\ref{fig:inits}.  The result
is shown in Figure~\ref{fig:densities}.  The densities are integrated over the
axis perpendicular to the orbital plane.

\begin{figure}
    \centering
    \includegraphics[width=0.9\linewidth]{figs/density_stars.png}
    \includegraphics[width=0.9\linewidth]{figs/density_darks.png}
    \caption{%
        Two-dimensional density histogram of the stellar (top three rows) and
        dark matter (bottom three rows) Sgr particles at various times for
        each considered merger.  Densities are integrated over the axis
        perpendicular to the orbital plane.
    }
    \label{fig:densities}
\end{figure}

The density plots allow us to more strictly quantify some of the trends noted
previously. For example, the dark matter density plots show quite concretely
that the dark matter distribution in the SIDM case peaks at the Galactic center,
where there is a distinct hole in the CDM mergers. The SIDM particles are also
significantly less well-constrained, with a large spread extending beyond the
limits of the plot. This is compared to the CDM cases, where there are
relatively few particles outside of the path of the stream.

Looking at the density distribution of the stellar particles allows us to
quantify some of the differences noted earlier.  For example, at 6.85 Gyr, the
SIDM stream has a relatively high-density ($\sim 10^5$ M$_\odot$/kpc$^2$)
stream arc extending from approximately $(-20,0)$ up to around $(60,125)$.
The corresponding arc exists in the CDM cases, but it is less well-defined
(there is more horizontal spread) and lower density ($\sim 10^4$
M$_\odot$/kpc$^2$).

There are other, similar differences, but we would do better to analyze these
differences for the specific time stamps of each merger which most closely
approximate Sgr today. Determining this requires mapping the position of the Sgr
progenitor, however.

\hypertarget{identifying-the-sgr-progenitor}{%
\section{Identifying the Sgr
progenitor}\label{identifying-the-sgr-progenitor}}

A key part of analyzing these data is to understand the trajectory and
evolution of the Sgr progenitor.  As such, we desire a method for successfully
identifying the position of the Sgr progenitor throughout its evolution.  This
is less straightforward than it may sound because of the strong effects of
tidal stripping.  These mean that we need to identify which particles are
stripped or bound to the progenitor at any given point and omit stripped
particles from our calculation of the progenitor position.  In our tests, we
tried a few different methods which we will describe here.

The first method that we tried was to track bound versus unbound star
particles by counting particles as stripped once they exceeded a fixed radius
from the center of mass of bound particles.  The algorithm for this is as
follows.  We begin by counting the stellar particles within a certain radius
on the first snapshot to be ``bound''.  For each snapshot after, we find the
center of mass of the bound particles.  Then, for each bound particle, we
compute its distance from the center of mass.  If this exceeds the fixed
stripping radius, we unmark the star as bound and continue.

As stated, this algorithm leaves has two parameters that can be tuned: the
initial stripping radius for the initial Sgr stellar positions and the fixed
stripping radius for all following snapshots. We found it useful to describe the
initial stripping radius instead in terms of the percentage of particles
that are initial counted as ``bound''. For example, we say that the we start
with the innermost 20\% of particles and proceed with a fixed radius of 20 kpc. 
effective for our mergers. The results of applying this algorithm to the
CDM/Cusp merger data with a few different choices of parameters can be seen in
Figure~\ref{fig:fixed_star}.

\begin{figure}
    \centering
    \includegraphics[width=0.9\linewidth]{figs/fixed_star.pdf}
    \caption{%
        Results of applying the ``fixed stripping radius''
        progenitor-identifying algorithm to the CDM/cusp merger data. Entries in
        the legend are given in the following format: ``a/b'' means that we
        started with the innermost ``a''\% of stellar particles and proceeded
        with a fixed stripping radius of ``b'' kpc.  In the upper left is the
        number of bound particles over time.  The upper right shows the
        stripping radius.  The bottom left shows the distance from the origin
        to the Sgr center of mass; an estimate of the MW-Sgr separation.  The
        bottom right shows the trajectory of the progenitor in the orbital
        plane.
    }
    \label{fig:fixed_star}
\end{figure}

This figure shows us that starting with too few of the initial particles (in
this case, 20\%) leads to very small numbers of bound particles at late times.
Further, it shows us that the trajectory of the progenitor can be somewhat
sensitive to the chosen algorithm parameters, especially at late times.

This algorithm appears to have two issues that we want to try to solve.
First, the actual size of the progenitor is expected to shrink with time, as
progressively more of the particles are stripped.  By using a fixed stripping
radius, we are not modeling the expected decay of the progenitor size.  The
second problem we encountered is that this method appears to leave us with
only $\mathcal{O}(10)$ bound particles after around 6 Gyr evolved.  As such,
we decided to explore modifications to the algorithm.

The first modification was to consider a decreasing stripping radius.  The
algorithm is very similar to before.  On the first snapshot, we count some
inner fraction of the particles to be ``bound'' and find the radius of this
ball of bound particles.  For the next snapshot, we strip any particles
which exceed this radius (times a constant).  \textit{However}, after
stripping away particles, we recompute the radius of the ball of the bound
particles, and reset the stripping radius equal to this. For each snapshot
following, we strip particles that are beyond this radius and recompute the
radius.  Over time, the stripping radius will decrease, modeling the
progressively decreasing size of the progenitor.  However, these modifications
tend to make the algorithm a bit \textit{too} aggressive, with it stripping
away all the particles in some cases.  We introduced a minimum stripping
radius such that the algorithm would choose the maximum of the given minimum and
the computed radius, but we found that the behavior in this case reduced to that
of the fixed-radius algorithm. The results of this decaying stripping radius are
shown in Figure~\ref{fig:decay_star}, with a minimum allowed stripping radius
of 8 kpc.

\begin{figure}
    \centering
    \includegraphics[width=0.9\linewidth]{figs/decay_star.pdf}
    \caption{%
        Results of applying the ``decaying stripping radius''
        progenitor-identifying algorithm to the CDM/cusp merger data with a
        minimum stripping radius of 8 kpc.  Plots have same meaning as in
        Figure~\ref{fig:fixed_star}.
    }
    \label{fig:decay_star}
\end{figure}

As can be seen in the figure, this algorithm leads to very aggressive stripping.
No matter the initial conditions, the desired stripping radius falls
\textit{very} quickly. For all considered parameter combinations, the desired
stripping radius fell below 8 kpc before the 6 Gyr mark, at which point the
algorithm will just default to using a fixed 8 kpc stripping radius. This
indicates that these changes are simply too aggressive and, when bundled with a
minimum stripping radius, is effectively equivalent to the fixed stripping
radius scheme.

One possible reason that it becomes too aggressive is because the actual
size of the progenitor is not monitonically decreasing.  Rather, its size
fluctuates over the course of the orbit, becoming quite compressed and small
near the pericenter and a bit more spread out and large near the apocenter.
These effects are modeled by the King formula for the tidal radius (todo cite
King) as given in~\cite{dierickx_predicted_2017}:
\begin{equation} \label{eq:king_radius}
    r_t = r \left[ \frac{1}{2} 
    \frac{M_{\text{Sgr}(<r_t)}}{M_{\text{MW}(<r)}} \right]^{1/3},
\end{equation}
where $M_{\text{gal}}(<r)$ is the enclosed halo mass in galaxy ``gal'' within
radius $r$ of the center of mass of the galaxy, $r$ is the distance between
the Milky Way and Sgr centers of mass, and $r_t$ is the tidal radius.  For a
given snapshot, then, we can compute the tidal radius according to this
formula by subtracting $r_t$ from both sides and using a simple root finder to
identify the $r_t$ which solves the equation.  We then strip any particles
which are farther than $r_t$ away from the center of mass of the progenitor at
the current time. The results of using the King tidal radius are shown in
Figure~\ref{fig:king_star}. Again, we pair this algorithm with a minimum
stripping radius of 8 kpc.

\begin{figure}
    \centering
    \includegraphics[width=0.9\linewidth]{figs/king_star.pdf}
    \caption{%
        Results of using the ``King tidal radius'' progenitor-identifying
        algorithm on the CDM/cusp merger data with a minimum stripping radius of
        8 kpc. The legend entries correspond to a fixed constant multiplied
        against the tidal radius before performing stripping. Plots again have
        the same meaning as in Figure~\ref{fig:fixed_star}.
    }
    \label{fig:king_star}
\end{figure}

Yet again, this algorithm appears to simply be too aggressive. In all cases, the
tidal radius quickly dips below the minimum stripping radius of 8 kpc. Further,
the algorithm appears to be somewhat unstable, defaulting to a recommended
radius of just \textit{zero} beyond a certain threshold. When this occurs, the
algorithm defaults to a fixed stripping radius equal to the minimum. Again,
then, this algorithm offers no improvements over the fixed radius.

At this point, it seems like the issue of too few bound particles at late times
may be a problem related to our relatively small number of Sgr stellar particles
instead of a problem with the algorithms themselves.  As such, we decided to
try to use the fixed algorithm but using both stellar \textit{and} dark matter
particles.  The result on the CDM/cusp data is shown in
Figure~\ref{fig:fixed_both}.

\begin{figure}
    \centering
    \includegraphics[width=0.9\linewidth]{figs/fixed_both.pdf}
    \caption{%
        Results of using the ``fixed stripping radius'' progenitor-identifying
        algorithm on the all particles in the CDM/cusp merger data. Plots and
        legend entries have the same meaning as in
        Figure~\ref{fig:fixed_star}.
    }
    \label{fig:fixed_both}
\end{figure}

This appears to yield promising results. We note that there are generally more
``bound'' particles at late times when using all particles than when only using
stellar ones, and that, aside from the ``50/20'' and ``80/20'' runs, the
resulting trajectories appear to be more robust to the algorithm parameters. As
such, we choose to move forward with this algorithm using the ``20/20''
parameters, as they appear to be consistent with the majority of the other
parameter choices and yield the most bound particles in the end. Using these
choices to identify the progenitor, we apply the algorithm to all three mergers.
The resulting data are shown in Figure~\ref{fig:all}.

\begin{figure}
    \centering
    \includegraphics[width=0.9\linewidth]{figs/all_mergers_pretty.pdf}
    \caption{%
        Results of using the ``fixed stripping radius'' progenitor-identifying
        algorithm on all the particles in each of the mergers. We use the
        ``20/20'' algorithm parameters for all mergers, meaning that we start
        with the inner 20\% of particles and use a 20 kpc stripping radius. The
        top row shows the trajectory of the Sgr progenitor in the orbital plane
        for each merger. The middle row plot shows the number of bound particles
        over time. The bottom row plot shows the MW-Sgr separation over time.
    }
    \label{fig:all}
\end{figure}

The plots in this figure are evidence that SIDM microphysics may indeed have a
profound impact on the resulting trajectory of the Sgr satellite.  One
difference can be see in that the number of bound particles decreases much
more substantially at early times than either of the two CDM runs.  This could
be explained by considering that self-interactions provide a mechanism for
dark matter to free itself from shallow gravitational potential wells to which
collisionless dark matter would remain confined.

Looking at the MW-Sgr separation over time, it is immediately evident that the
cored profile yields a slightly longer orbital period, given the slowly
increasing distance between the apo- and pericenters of the CDM/cusp and
CDM/core orbits.  The inclusion of self-interactions appears to add to this
effect, with the CDM/cusp attaining six pericenters before the SIDM orbit is
able to reach a fifth.

One phenomenon showcased by the separation curves which is difficult to
understand is the lack of a consistent decay in the apocenters. We compare to
figure 6 of~\cite{dierickx_predicted_2017}, which shows a steadily decreasing
pericenter.  This is the expected behavior; as the Sgr progenitor orbits and
decays, we would expect it to lose energy and steadily fall inward.  This,
however, does not appear in our plots.  In fact, this phenomenon is
exacerbated in the SIDM merger, where the apocenter actually appears to
\textit{grow} after around 4 Gyr, reaching nearly 130 kpc at the 6 Gyr mark.
The mechanism by which this would occur is not yet understood.

\hypertarget{time-of-closest-match}{%
\section{Time of closest match}\label{time-of-closest-match}}

As stated previously, a description of the orbit of the progenitor will allow us
to determine the specific time stamps at which our mergers most closely
approximate Sgr today.  We take the observed coordinates of Sgr to be
$(\alpha, \delta) = (283.83, -29.45)$ (where $\alpha$ is right ascension and
$\delta$ is declination) degrees~\cite{nasa_nasaipac_nodate}, proper motion
$(\mu_\alpha \cos\delta, \mu_\delta) = (-2.54 \pm 0.18, -1.19 \pm 0.16)$
mas/yr~\cite{massari_hubble_2013}, distance $24.8 \pm 0.8$
kpc~\cite{kunder_distance_2009}, and line-of-sight velocity $179 \pm 1$
km/s~\cite{dierickx_predicted_2017,bellazzini_nucleus_2008}. We can then convert
the coordinates of the Sgr progenitor to equatorial coordinates and look at the
evolution over time, comparing against these reference values. The result is
shown in Figure~\ref{fig:eq_prog}.

\begin{figure}
    \centering
    \includegraphics[width=1.0\linewidth]{figs/equatorial_progenitor.pdf}
    \caption{%
        Trajectory of the Sgr progenitor in terms of equatorial coordinates and
        proper motions for each merger. Black squares represent the observed
        coordinates of the progenitor today.
    }
    \label{fig:eq_prog}
\end{figure}

We can also focus specifically on the position of the progenitor instead of its
velocity, looking specifically at the right ascension, declination, and
distance. We find the orbit to be strongly period in right ascension, so these
plots have been ``unwrapped'' to better show the evolution of the progenitor
over time. The results are shown in Figure~\ref{fig:eq_prog_unwrapped}.

\begin{figure}
    \centering 
    \includegraphics[width=0.9\linewidth]{figs/equatorial_progenitor_coords_unwrapped.pdf}
    \caption{%
        Trajectory of the simulated Sgr progenitor in terms of positional
        coordinates only. The orbit is periodic in right ascension, so the
        coordinates have been ``unwrapped'' to better show the behavior.
        Observed values of the right ascension, declination, and distance are
        shown as dashed lines.
    }
    \label{fig:eq_prog_unwrapped}
\end{figure}

We note that the Dierickx 2017 model~\cite{dierickx_predicted_2017} most closely
approximated the observed coordinates just after its fifth pericenter. We note
that no snapshot of our model comes as close to these coordinates as theirs, but
we believe that this discrepancy would be alleviated somewhat by greater time
resolution in the snapshots of the stream.



\hypertarget{comparison-to-stream-data}{%
\section{Comparison to stream data}\label{comparison-to-stream-data}}

The last piece of the analysis that we will consider is a comparison to observed
data.  The data we will compare against comes from the second data release
(DR2) of the \textit{Gaia}
mission~\cite{lindegren_gaia_2018,gaia_collaboration_gaia_2018}, from which Sgr
stars have been identified by Ibata et al.~\cite{ibata_panoramic_2020} using the
\verb|STREAMFINDER|
algorithm~\cite{malhan_streamfinder_2018,malhan_ghostly_2018}. The resulting
dataset includes the \textit{Gaia} equatorial coordinates, proper motions,
magnitudes, and colors of 263,438 stars, along with an estimate of the distance
provided by the algorithm. They find their dataset to agree well with the Law
and Majewski 2010 model~\cite{law_sagittarius_2010}, barring a few small
deviations. The \verb|STREAMFINDER| sample is shown in Figure~\ref{fig:streamfinder}.

\begin{figure}
    \centering 
    \includegraphics[width=0.7\linewidth]{figs/streamfinder.pdf}
    \caption{%
        Equatorial coordinates and estimated distances for the 263,438 Sgr
        stars identified by the \texttt{STREAMFINDER} algorithm.
    }
    \label{fig:streamfinder}
\end{figure}

In the Dierickx 2017 model~\cite{dierickx_predicted_2017}, the simulation was
found to most closely approximate the position of Sgr today at the fifth
pericenter.  As such, we consider the corresponding times in our mergers.  In
our mergers, we find the fifth pericenter to occur after 7.0 Gyr in the
CDM/cusp merger, 7.4 Gyr in the CDM/core merger, and 8.8 Gyr in the SIDM
merger.

For each merger at the corresponding times, we convert the coordinates of the
stellar particles into equatorial coordinates in the ICRS frame using the
Astropy
package~\cite{astropy_collaboration_astropy_2013,astropy_collaboration_astropy_2018}.
We only consider stars with distances less than 100 kpc, as this is done
in~\cite{dierickx_predicted_2017} and the \verb|STREAMFINDER| sample contains no
stars beyond this threshold.  The resulting coordinates are plotted in
Figure~\ref{fig:equatorial}, with the \verb|STREAMFINDER| density plot shown
in gray.

\begin{figure}
    \centering 
    \includegraphics[width=1.0\linewidth]{figs/equatorial_streamfinder.png}
    \caption{%
        Equatorial coordinates for our simulated Sgr stellar particles of
        distances less than 100 kpc at the time of the fifth pericenter for
        each merger.  The position of the progenitor is shown by a black circle.
        The density of STREAMFINDER stars is also shown in gray.
    }
    \label{fig:equatorial}
\end{figure}

Our comparison to the \verb|STREAMFINDER| stars shows qualitatively good
agreement in terms of the right ascension and declination. The thickness of the
stream appears to be narrower in some places, and our maxmium achieved
declination is approximately 60$^\circ$ where the \verb|STREAMFINDER| data
barely exceeds 40$^\circ$. We also see qualitative agreement for the primary
stream arms in terms of distance, though our stream appears to be shifted
relative to the \verb|STREAMFINDER| data. Our model also predicts an extension
to the Sgr stream arms, similar to the results of Dierickx et
al.~\cite{dierickx_predicted_2017}. 

There also exist some differences between the three mergers in this frame.
For example, the CDM/cusp run appears to have more particles with larger
line-of-sight velocity magnitudes, with more stars that appear to be dark blue
and red.  Inversely, the SIDM merger appears to have much more limited
variation in line-of-sight velocity.  Also, the SIDM and CDM runs appear to
have different features which are well-defined; the CDM runs show a rising
stream arm at right ascension $\approx 350^\circ$ that is not well-represented
in the SIDM run.  The SIDM run, however, has a more well-defined falling
stream arm at right ascension $\approx 200^\circ$.

To more fully explore these distances, we can look at histograms of the stellar
particles' equatorial coordinates and proper motions and compare to the same for
the \verb|STREAMFINDER| stars. We note, however, the presence of a very dense
region of stars in the \verb|STREAMFINDER| data at $\text{RA} = 290^\circ$,
$\text{Dec} = -30^\circ$, and $\text{dist} = 30$ kpc. This greatly skews the
\verb|STREAMFINDER| distributions, so we cut it off in the plots in order to
more fully elucidate the simulation distributions. The resulting histograms may
be seen in Figure~\ref{fig:eq_hists}.

\begin{figure}
    \centering
    \includegraphics[width=0.9\linewidth]{figs/equatorial_hists.pdf}
    \caption{%
        Histograms of the equatorial coordinates and proper motions of the
        simulated stellar particles closer than 100 kpc and the reference
        \texttt{STREAMFINDER} stars (where possible).  Note that the
        \texttt{STREAMFINDER} sample contains a very dense region which is
        intentionally cut off in the right ascension, declination, and
        distance plots in order to better see the simulated distributions.
        All distributions are normalized to unity.
    }
    \label{fig:eq_hists}
\end{figure}

The histograms show us that the distributions of simulated stars are quite
different than the \verb|STREAMFINDER| sample, particularly when we consider
the proper motions and distances. Specifically, the $\mu_\alpha \cos\delta$
distribution is largely concentrated between $-1$ and $0.5$ mas/yr for all
three mergers, but appears to be a roughly unimodal distribution at $-2.5$
mas/yr in the \verb|STREAMFINDER| data.  Similarly, all three mergers show a
strong peak in $\mu_\delta$ at approximately $-0.5$ mas/yr, compared to the
strong peak at $-1.5$ mas/yr in the \verb|STREAMFINDER| set. In the distance
distribution, we see the distribution of distances appear to rise until around
60 to 80 kpc, compared to the sharp peak around 30 kpc in the
\verb|STREAMFINDER| set with a decreasing frequency for larger distances.

The declination plot shows the existence of peaks around $-30^\circ$ to
$-40^\circ$ for all mergers and the \verb|STREAMFINDER| set. However, the
mergers \textit{also} have a peak at around $+20^\circ$ that appears to be
largely non-existent in \verb|STREAMFINDER|.  Lastly, we note that the right
ascension plot shows some similarities between all mergers and the
\verb|STREAMFINDER|: particularly, a peak at around $225^\circ$ and a trough
at around $140^\circ$.

The histograms also allow us to highlight some differences between the mergers
themselves. In particular, the radial velocity histogram confirms the previously
noted trend. The CDM/cusp merger appears to have more particles with higher
radial velocity magnitudes than either of the other runs. The CDM/core merger
appears to have a much flatter distribution of radial velocities than either of
the other two mergers. The SIDM merger, on the other hand, has a strongly
bimodal distribution of radial velocities, with strong peaks around $+100$ and
$-150$ km/s.

