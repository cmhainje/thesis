\hypertarget{conclusions}{%
\chapter{Conclusions}\label{conclusions}}

% highlighting future to-dos and refinements, and identifying one or two key
% observations that should be pursued more carefully

In this work, we have presented a novel test of self-interacting dark matter
models by studying simulations of the infall of the Sagittarius dwarf
spheroidal galaxy.  Following the Dierickx 2017~\cite{dierickx_predicted_2017}
model, we have performed three simulations of the infall of Sgr.  Two of these
used the CDM model, one using initially cuspy halos and the other using cored
halos.  The third used the SIDM model, starting from cored halo profiles and
assuming a cross section of $\sigma/m = 10$ cm$^2$/g.

In our analysis of the resulting evolutions, we found that all three develop a
strong, streamlike shape.  The inclusion of self-interactions, however,
changes the shape dramatically, seemingly changing the overall rotation of the
outer stream arms and causing the loss of the intricate inner structure seen
in both CDM simulations.  In general, it appears that the SIDM simulation was
significantly more susceptible to the effects of tidal stripping.

Comparisons to existing Sgr data show generally good agreement especially with
our CDM streams.  For example, the stream shape seen in
\verb|STREAMFINDER| \cite{ibata_panoramic_2020} data is reproduced quite
faithfully in equatorial coordinates with similar distributions in
heliocentric distance.  Our mergers also show the predicted extension to the
stream introduced by the Dierickx 2017 model.  The SIDM stream, while
qualitatively similar to the \verb|STREAMFINDER| shape, shows the poorest
agreement of our three simulations with significantly larger declinations than
expected.  Similar findings were shown with a comparison to the right
ascension and declination data of M-giant stars from
2MASS~\cite{majewski_two_2003}.

Moreover, the resulting SIDM progenitor is much slower than either of the CDM
progenitors.  Its orbital time lags significantly, with the SIDM progenitor
reaching its fifth pericentric passage \textit{after} the cuspy CDM progenitor
is able to reach its sixth.  This, we believe, is intimately tied to
significantly smaller line-of-sight velocities at late times.  The SIDM
progenitor is in fact \textit{not} able to reproduce the observed heliocentric
distance and line-of-sight velocity of the Sgr progenitor at later times when
the stream is sufficiently developed in our simulation. Similarly, the SIDM
stream gives the least accurate reproduction of the heliocentric distance and
line-of-sight velocities of the 2MASS data.

These findings suggest that line-of-sight velocity in particular may serve as
an important discriminator for the existence of dark matter self-interaction.
However, there are many caveats with our findings that must be more thoroughly
examined in future studies. The first and most obvious is that we have only
considered a single, large, velocity-independent cross section of $\sigma / m =
10$ cm$^2$/g. Future studies should consider a wider range of cross sections, in
particular those close to 1 cm$^2$/g, and should consider adding
velocity-dependence to better match observational constraints. 

The next caveat is that we have only considered initial simulation parameters
(such as the initial position and velocity of Sgr) which were derived as
best-fit parameters for a CDM simulation. However, the optimal description of
the Sgr orbital history with SIDM likely differs from that of the CDM
description.  As such, the development of a semi-analytic model to explore a
wider range of parameters, as was done in~\cite{dierickx_predicted_2017}, would
help to identify such parameters and reduce any differences caused by this.

Future studies would also do well to improve the resolution of these
simulations.  Saving snapshots of the simulation at smaller time intervals
would significantly improve the ability to match the progenitor coordinates to
observation, which is currently time resolution-limited.  Using more stellar
particles would also likely improve the details in the resulting streams.  We
note further that the inclusion of a bulge in the galaxy profiles and the use
of other progenitor-identifying algorithms would be greatly helpful for more
accurately describing the trajectory of the simulated progenitors.

We note finally that the compilation of radial velocity data for more stars in
the Sgr stream would largely improve the comparisons we make with our mergers.
The \verb|STREAMFINDER| team note that nearly 3,000 stars in their stream have
radial velocity information in other public surveys; a compilation of these
data would help expand the comparisons that were made at the end of this work
and confirm whether the velocity distributions seen in our models are
observed.
